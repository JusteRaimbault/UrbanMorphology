\documentclass[12pt]{article}

\usepackage{times}

% general packages without options
\usepackage{amsmath,amssymb,bbm}
% graphics
\usepackage{graphicx}
% text formatting
\usepackage[document]{ragged2e}
\usepackage{pagecolor,color}

\newcommand{\noun}[1]{\textsc{#1}}

\usepackage[utf8]{inputenc}
\usepackage[T1]{fontenc}
% geometry
\usepackage[margin=2.5cm]{geometry}

\usepackage{multicol}
\usepackage{setspace}

\usepackage{natbib}
\setlength{\bibsep}{0.0pt}

%\usepackage[french]{babel}

% layout : use fancyhdr package
%\usepackage{fancyhdr}
%\pagestyle{fancy}

% variable to include comments or not in the compilation ; set to 1 to include
\def \draft {1}


% writing utilities

% comments and responses
%  -> use this comment to ask questions on what other wrote/answer questions with optional arguments (up to 4 answers)
\usepackage{xparse}
\usepackage{ifthen}
\DeclareDocumentCommand{\comment}{m o o o o}
{\ifthenelse{\draft=1}{
    \textcolor{red}{\textbf{C : }#1}
    \IfValueT{#2}{\textcolor{blue}{\textbf{A1 : }#2}}
    \IfValueT{#3}{\textcolor{ForestGreen}{\textbf{A2 : }#3}}
    \IfValueT{#4}{\textcolor{red!50!blue}{\textbf{A3 : }#4}}
    \IfValueT{#5}{\textcolor{Aquamarine}{\textbf{A4 : }#5}}
 }{}
}

% todo
\newcommand{\todo}[1]{
\ifthenelse{\draft=1}{\textcolor{red!50!blue}{\textbf{TODO : \textit{#1}}}}{}
}


\makeatletter


\makeatother


\linespread{1.25}

\begin{document}







\title{
A multi-dimensional percolation approach to characterize sustainable mega-city regions
\medskip\\
\textit{Marami 2018
%Communication proposal
}
}
\author{\noun{Juste Raimbault}\medskip\\
(UPS CNRS 3611 ISC-PIF et UMR CNRS 8504 Géographie-cités)\\
}
\date{}

\maketitle

\justify

\pagenumbering{gobble}


\textbf{Keywords : }\textit{Road network; multilayer percolation; mega-city region}

\medskip

The structure of road networks both translates its past growth dynamics and has a significant impact on the sustainability of territories it irrigates. A method to characterize topologies of these spatial networks is network percolation. Such approaches have been applied to the modeling of urban growth \citep{makse1998modeling} and to the analysis of street networks for example to extract endogenous urban regions \citep{arcaute2016cities} or to characterize the spatial morphology of point patterns \cite{huynh2018characterisation}. Existing heuristics however generally focus on a single morphological dimension of networks, and leave out the functional properties of urban systems \citep{burger2012form}.

This communication addresses such a gap by introducing a multi-dimensional percolation heuristic, which is analog to multilayer network percolation \citep{boccaletti2014structure}. Given discrete spatial fields, site percolation is operated between two cells given a threshold parameter for each dimension and a distance threshold. We apply the heuristic to urban morphology and road network topology measures in Europe. More precisely, a grid with resolution 50km of population density morphology indicators and road network topology indicators, has been computed on spatial moving windows for all European Union by \cite{raimbault2018urban}. We percolate the population density layer with a network characteristic layer, that we test among number of edges, number of vertices, cyclomatic number and euclidian efficiency, which capture functional properties especially for the two last.

We systematically explore the clusters obtained for 840 parameter configurations. Maps reveal that most configurations resemble the actual distribution of European mega-city regions, which are functionally integrated polycentric urban areas \citep{hall2006polycentric}. We use this endogenous definition of regional urban systems the percolation algorithm produced to evaluate their sustainability, in terms of conflicting objectives of economic integration and greenhouse gases emissions. Applying a gravity model to each region, we estimate transportation flows within each and extrapolate emissions by coupling with the Edgar emission database \citep{janssens2017edgar} and economic activities with a scaling law of population. We show therein that different population, network and distance thresholds yield different performances in terms of sustainability, exhibiting a Pareto front. This suggests policies in terms of regional integration to increase the sustainability of mega-city regions. Further work will consist in the use of calibration heuristics \citep{reuillon2013openmole} to find in a more robust way optimal parameter values.





%%%%%%%%%%%%%%%%%%%%
%% Biblio
%%%%%%%%%%%%%%%%%%%%
%\tiny

%\begin{multicols}{2}

%\setstretch{0.3}
%\setlength{\parskip}{-0.4em}


\bibliographystyle{apalike}
\bibliography{biblio}
%\end{multicols}



\end{document}
