\documentclass{jimis-en}
%DIF LATEXDIFF DIFFERENCE FILE
%DIF DEL paper_v1.tex   Wed Sep 11 16:06:20 2019
%DIF ADD paper.tex      Wed Sep 11 20:25:32 2019
\pdfoutput=1 %forces arXiv/HAL to compile with PDFLaTeX

% Packages
\usepackage{array}
\usepackage{pgfplots}
\usepackage{graphicx}

% header
\title{Multi-dimensional Urban Network Percolation}
% Identifying European Sustainable Mega-city Regions with Multi-dimensional Network Percolation
% Endogenous urban regions as an optimal compromise between economic integration and greenhouse gases emissions
% A multi-dimensional percolation approach to characterize sustainable mega-city regions



\author[1,2,3,*]{Juste RAIMBAULT}
\affil[1]{UPS CNRS 3611 ISC-PIF, France} 
\affil[2]{CASA, UCL, UK}
\affil[3]{UMR CNRS 8504 G{\'e}ographie-cit{\'e}s}
 
% corresponding author with his/her email
\corrauthor{juste.raimbault@polytechnique.edu}

%% Information provided by JIMIS:
%ISSN
% DOI 
\doi{10.18713/JIMIS-ddmmyy-v-a}{}
% reviewing process: first date = submission, second date = acceptance 
\review{Month-in-letters Day Year}{Month-in-letters Day Year}
% Volume and Year of the issue
\publication{N}{YYYY}

%% Information provided by the Editors:
% Issue Title 
\issue{Title of the interdisciplinary issue}
% List of k Editors
\editors{}
%DIF PREAMBLE EXTENSION ADDED BY LATEXDIFF
%DIF UNDERLINE PREAMBLE %DIF PREAMBLE
\RequirePackage[normalem]{ulem} %DIF PREAMBLE
\RequirePackage{color}\definecolor{RED}{rgb}{1,0,0}\definecolor{BLUE}{rgb}{0,0,1} %DIF PREAMBLE
\providecommand{\DIFadd}[1]{{\protect\color{blue}\uwave{#1}}} %DIF PREAMBLE
\providecommand{\DIFdel}[1]{{\protect\color{red}\sout{#1}}}                      %DIF PREAMBLE
%DIF SAFE PREAMBLE %DIF PREAMBLE
\providecommand{\DIFaddbegin}{} %DIF PREAMBLE
\providecommand{\DIFaddend}{} %DIF PREAMBLE
\providecommand{\DIFdelbegin}{} %DIF PREAMBLE
\providecommand{\DIFdelend}{} %DIF PREAMBLE
%DIF FLOATSAFE PREAMBLE %DIF PREAMBLE
\providecommand{\DIFaddFL}[1]{\DIFadd{#1}} %DIF PREAMBLE
\providecommand{\DIFdelFL}[1]{\DIFdel{#1}} %DIF PREAMBLE
\providecommand{\DIFaddbeginFL}{} %DIF PREAMBLE
\providecommand{\DIFaddendFL}{} %DIF PREAMBLE
\providecommand{\DIFdelbeginFL}{} %DIF PREAMBLE
\providecommand{\DIFdelendFL}{} %DIF PREAMBLE
\newcommand{\DIFscaledelfig}{0.5}
%DIF HIGHLIGHTGRAPHICS PREAMBLE %DIF PREAMBLE
\RequirePackage{settobox} %DIF PREAMBLE
\RequirePackage{letltxmacro} %DIF PREAMBLE
\newsavebox{\DIFdelgraphicsbox} %DIF PREAMBLE
\newlength{\DIFdelgraphicswidth} %DIF PREAMBLE
\newlength{\DIFdelgraphicsheight} %DIF PREAMBLE
% store original definition of \includegraphics %DIF PREAMBLE
\LetLtxMacro{\DIFOincludegraphics}{\includegraphics} %DIF PREAMBLE
\newcommand{\DIFaddincludegraphics}[2][]{{\color{blue}\fbox{\DIFOincludegraphics[#1]{#2}}}} %DIF PREAMBLE
\newcommand{\DIFdelincludegraphics}[2][]{% %DIF PREAMBLE
\sbox{\DIFdelgraphicsbox}{\DIFOincludegraphics[#1]{#2}}% %DIF PREAMBLE
\settoboxwidth{\DIFdelgraphicswidth}{\DIFdelgraphicsbox} %DIF PREAMBLE
\settoboxtotalheight{\DIFdelgraphicsheight}{\DIFdelgraphicsbox} %DIF PREAMBLE
\scalebox{\DIFscaledelfig}{% %DIF PREAMBLE
\parbox[b]{\DIFdelgraphicswidth}{\usebox{\DIFdelgraphicsbox}\\[-\baselineskip] \rule{\DIFdelgraphicswidth}{0em}}\llap{\resizebox{\DIFdelgraphicswidth}{\DIFdelgraphicsheight}{% %DIF PREAMBLE
\setlength{\unitlength}{\DIFdelgraphicswidth}% %DIF PREAMBLE
\begin{picture}(1,1)% %DIF PREAMBLE
\thicklines\linethickness{2pt} %DIF PREAMBLE
{\color[rgb]{1,0,0}\put(0,0){\framebox(1,1){}}}% %DIF PREAMBLE
{\color[rgb]{1,0,0}\put(0,0){\line( 1,1){1}}}% %DIF PREAMBLE
{\color[rgb]{1,0,0}\put(0,1){\line(1,-1){1}}}% %DIF PREAMBLE
\end{picture}% %DIF PREAMBLE
}\hspace*{3pt}}} %DIF PREAMBLE
} %DIF PREAMBLE
\LetLtxMacro{\DIFOaddbegin}{\DIFaddbegin} %DIF PREAMBLE
\LetLtxMacro{\DIFOaddend}{\DIFaddend} %DIF PREAMBLE
\LetLtxMacro{\DIFOdelbegin}{\DIFdelbegin} %DIF PREAMBLE
\LetLtxMacro{\DIFOdelend}{\DIFdelend} %DIF PREAMBLE
\DeclareRobustCommand{\DIFaddbegin}{\DIFOaddbegin \let\includegraphics\DIFaddincludegraphics} %DIF PREAMBLE
\DeclareRobustCommand{\DIFaddend}{\DIFOaddend \let\includegraphics\DIFOincludegraphics} %DIF PREAMBLE
\DeclareRobustCommand{\DIFdelbegin}{\DIFOdelbegin \let\includegraphics\DIFdelincludegraphics} %DIF PREAMBLE
\DeclareRobustCommand{\DIFdelend}{\DIFOaddend \let\includegraphics\DIFOincludegraphics} %DIF PREAMBLE
\LetLtxMacro{\DIFOaddbeginFL}{\DIFaddbeginFL} %DIF PREAMBLE
\LetLtxMacro{\DIFOaddendFL}{\DIFaddendFL} %DIF PREAMBLE
\LetLtxMacro{\DIFOdelbeginFL}{\DIFdelbeginFL} %DIF PREAMBLE
\LetLtxMacro{\DIFOdelendFL}{\DIFdelendFL} %DIF PREAMBLE
\DeclareRobustCommand{\DIFaddbeginFL}{\DIFOaddbeginFL \let\includegraphics\DIFaddincludegraphics} %DIF PREAMBLE
\DeclareRobustCommand{\DIFaddendFL}{\DIFOaddendFL \let\includegraphics\DIFOincludegraphics} %DIF PREAMBLE
\DeclareRobustCommand{\DIFdelbeginFL}{\DIFOdelbeginFL \let\includegraphics\DIFdelincludegraphics} %DIF PREAMBLE
\DeclareRobustCommand{\DIFdelendFL}{\DIFOaddendFL \let\includegraphics\DIFOincludegraphics} %DIF PREAMBLE
%DIF END PREAMBLE EXTENSION ADDED BY LATEXDIFF

\begin{document}

\maketitle

\DIFdelbegin %DIFDELCMD < \abstract{Network percolation has recently been proposed as a method to characterize the global structure of an urban system form the bottom-up. This paper proposes to extend urban network percolation in a multi-dimensional way, to take into account both urban form (spatial distribution of population) and urban functions (here as properties of transportation networks). The method is applied to the European urban system to reconstruct endogenous urban regions. The variable parametrization allows to consider patterns of optimization for two stylized contradictory sustainability indicators (economic performance and greenhouse gases emissions). This suggests a customizable spatial design of policies to develop sustainable territories.}
%DIFDELCMD < %%%
\DIFdelend \DIFaddbegin \abstract{Network percolation has recently been proposed as a method to characterize the hierarchical structure of an urban system form the bottom-up. This paper proposes to extend urban network percolation in a multi-dimensional way, to take into account both urban form (spatial distribution of population) and urban functions (here as properties of transportation networks). The method is applied to the European urban system to reconstruct endogenous urban regions. The variable parametrization allows to consider patterns of optimization for two stylized contradictory sustainability indicators (economic performance and greenhouse gases emissions). This suggests a customizable spatial design of policies to develop sustainable territories.}
\DIFaddend 

\keywords{Road network; Multi-dimensional percolation; European urban system; Mega-city region}



\DIFdelbegin %DIFDELCMD < \added{\cite{aleta2018multilayer} multiplex/multilayer}
%DIFDELCMD < 

%DIFDELCMD < %%%
\DIFdelend \section{Introduction}

\strut
\vspace{-4ex}




\DIFdelbegin %DIFDELCMD < \added{traffic : other scale \cite{Li669} \cite{Zeng23} }
%DIFDELCMD < %%%
\DIFdelend \DIFaddbegin \subsection{Towards multidimensional urban network percolation}
\DIFaddend 

The structure of road networks can be used as a proxy to understand its past growth dynamics, but also has a significant impact on the future sustainability of territories it irrigates. Diverse methods to characterize the structure of spatial networks, and more particularly road networks, have been developed in that context\DIFdelbegin \DIFdel{, including }\DIFdelend \DIFaddbegin \DIFadd{. They include }\DIFaddend classical network indicators such as centralities \citep{crucitti2006centrality} but also more elaborated constructions capturing more realistic processes in terms of street network use \citep{lagesse2015spatial}. \DIFdelbegin \DIFdel{These study }\DIFdelend \DIFaddbegin \DIFadd{Such studies of urban networks }\DIFaddend are by essence interdisciplinary, or at least imply complementary viewpoints from \DIFdelbegin \DIFdel{disciplines as diverse as }\DIFdelend \DIFaddbegin \DIFadd{diverse disciplines. These for example include }\DIFaddend architecture with space syntax \citep{hillier1976space}, physics with the study of spatial networks \citep{barthelemy2011spatial}, or social science disciplines concerned with space such as geography \citep{ducruet2014spatial}.


A method to characterize \DIFdelbegin \DIFdel{topologies }\DIFdelend \DIFaddbegin \DIFadd{the hierarchical structure }\DIFaddend of such urban spatial networks is network percolation, initially applied to \DIFaddbegin \DIFadd{urban }\DIFaddend road networks by \cite{arcaute2016cities}. Percolation in physics can be understood in a broad sense as processes related to the progressive occupation or connection of nodes of a network\DIFdelbegin \DIFdel{, and }\DIFdelend \DIFaddbegin \DIFadd{. It }\DIFaddend is generally associated to a phase transition with the emergence of a giant cluster at a given \DIFdelbegin \DIFdel{connection probability \mbox{%DIFAUXCMD
\citep{stauffer2014introduction}}%DIFAUXCMD
. Important applications }\DIFdelend \DIFaddbegin \DIFadd{probability of connection \mbox{%DIFAUXCMD
\citep{stauffer2014introduction}}%DIFAUXCMD
. Practical applications in different fields }\DIFaddend include the quantification of network robustness \citep{callaway2000network} or the modeling of epidemic spreading \citep{newman1999scaling}.
\DIFaddbegin 

\DIFaddend Such approaches have been applied to urban systems \DIFdelbegin \DIFdel{not only for }\DIFdelend \DIFaddbegin \DIFadd{with other applications than }\DIFaddend the study of networks. \cite{makse1998modeling} model urban growth with a local percolation model for site occupancy. \cite{arcaute2016cities} focus on the analysis of street networks and extract endogenous urban regions for UK which correlate with socio-economic properties, and provide a definition of urban areas which highly correlates with land-cover data. \cite{piovani2017urban} apply road network percolation at the mesoscopic scale of London metropolitan area, in relation with a retail location model. \DIFaddbegin \DIFadd{At a larger scale, the paradigm of percolation transition has been applied to the study of urban traffic dynamics \mbox{%DIFAUXCMD
\citep{Li669,Zeng23}}%DIFAUXCMD
. }\DIFaddend In spatial statistics, this method can be used to characterize the spatial morphology of point patterns \citep{huynh2018characterisation}.


Existing heuristics \DIFdelbegin \DIFdel{however }\DIFdelend generally focus on a single dimension or property of the urban system. However, such systems are known to be \DIFdelbegin \DIFdel{multidimensional, for examplethrough the superposition of the }\DIFdelend \DIFaddbegin \DIFadd{highly multidimensional. For example, the }\DIFaddend morphological dimension of networks \DIFdelbegin \DIFdel{, and }\DIFdelend \DIFaddbegin \DIFadd{is complementary to }\DIFaddend the functional properties of the urban environment \citep{burger2012form}. The link between urban form and function remains in particular an open question \citep{batty1994fractal}\DIFdelbegin \DIFdel{, but more generally}\DIFdelend \DIFaddbegin \DIFadd{. More generally, }\DIFaddend the inclusion of multiple dimension in urban analysis is still a research direction to be investigated, as in the case of agent-based models for example \citep{perez2016agent}. This paper addresses such a gap in the case of urban network percolation, by introducing a multi-dimensional percolation heuristic. The method allows \DIFdelbegin \DIFdel{to combine }\DIFdelend \DIFaddbegin \DIFadd{combining }\DIFaddend different dimensions of the urban system, the same way that \cite{cottineau2018defining} \DIFdelbegin \DIFdel{combines population density and }\DIFdelend \DIFaddbegin \DIFadd{link population density with }\DIFaddend commuting flows to produce multiple definitions of urban areas.


\DIFaddbegin \subsection{Sustainability of mega-urban regions}


\DIFaddend Beside these methodological issues \DIFdelbegin \DIFdel{, applied }\DIFdelend \DIFaddbegin \DIFadd{of characterizing urban networks and more particularly their endogenous hierarchical structure, some related applied research issues can be considered. Indeed, quantitative }\DIFaddend tools are needed to \DIFdelbegin \DIFdel{quantify }\DIFdelend \DIFaddbegin \DIFadd{evaluate }\DIFaddend the sustainability of recently emerged urban forms. In particular, according to \cite{lenechet2017peupler}, the most recent transition of human settlement systems (in the sense of \cite{sanders2017peupler}, i.e. a change in the dynamical regime ruling the evolution of the spatial structure of settlements) is the emergence of mega-city regions. These have been defined by \cite{hall2006polycentric} as polycentric urban structures highly integrated in terms of flows. The transition imply complex processes such as changes in the governance structure, and can not be associated to the stylized transition identified by \cite{louf2013modeling} in a \DIFdelbegin \DIFdel{simplistic }\DIFdelend toy urban model \DIFdelbegin \DIFdel{, and therefore has chances to imply more drivers than }\DIFdelend \DIFaddbegin \DIFadd{including }\DIFaddend negative externalities of congestion only.
\DIFaddbegin 

\DIFaddend To what extent these new urban forms are sustainable, for example in the broad sense of UN development goals \citep{komiyama2006sustainability}, remains an open question. \DIFdelbegin \DIFdel{In order to test our }\DIFdelend \DIFaddbegin \DIFadd{Indeed, these integrated mega-city regions may for example imply different patterns of economic and transportation flows and thus exhibit various performances regarding different indicators of sustainability.
}

\DIFadd{Case studies of targeted megacities have for example focused on the links between urban form and mobility or resources management \mbox{%DIFAUXCMD
\citep{sorensen2010megacities}}%DIFAUXCMD
. A significant literature has focused on the sustainability of mega urban regions with a qualitative approach, such as \mbox{%DIFAUXCMD
\cite{laquian2011planning} }%DIFAUXCMD
which establish a typology of these and corresponding governance structures in the Asian context. In the case of Europe, \mbox{%DIFAUXCMD
\cite{marull2013emerging} }%DIFAUXCMD
use econometric analysis to see the economic and ecological advantage of integrated urban regions. \mbox{%DIFAUXCMD
\cite{feng2018spatiotemporal} }%DIFAUXCMD
introduce a method to measure the level of polycentricity of an urban mega-region. \mbox{%DIFAUXCMD
\cite{su2017china} }%DIFAUXCMD
propose a framework to evaluate the performance of urban mega-regions, regarding economic, environmental, social and spatial dimensions.
}

\DIFadd{Beside these targeted studies, a remaining open issue, to the best of our knowledge never tackled at this small geographical scale, is the endogenous characterization of such urban regions. The delineation of these geographical systems is often taken as exogenous, and their performance and sustainability is then evaluated. Several approaches can be taken to define such systems, including integration through transportation networks, continuity of night lights, and economic productivity thresholds~\mbox{%DIFAUXCMD
\citep{lang2005beyond,florida2008rise}}%DIFAUXCMD
. In these existing studies, parameters to define such regions remain fixed, and an endogenous hierarchical structure is possibly ignored. The second component of our research question will thus be the endogenous characterization of urban regions and how this can be applied to study their sustainability.
}



\subsection{Proposed approach}

\DIFadd{This work proposes to partly tackle the two above research questions by linking them. More precisely, it first investigates from a methodological viewpoint how urban network percolation can be generalized to multiple dimensions, and secondly explores the endogenous characterization of mega-city regions using such a }\DIFaddend multi-dimensional percolation\DIFdelbegin \DIFdel{method, we propose to apply }\DIFdelend \DIFaddbegin \DIFadd{, and how this can be used to quantify the sustainability of these systems. These two axis are tightly linked. Indeed, on the one hand street network percolation has initially been proposed to identify endogenous entities in territorial systems, and on the other hand mega-urban regions are characterized simultaneously by morphological dimensions (continuity of the built environment) and functional dimensions (high integration of flows).
}

\DIFadd{We therefore consider percolation on two dimensions of the urban systems, one linked to urban form which is the spatial distribution of population, and one linked to transportation networks, through structural indicators quantifying local road networks. We apply in particular the heuristic to urban morphology and road network topology measures in Europe. The idea to combine urban form with network topology measures relies on the capture of the link between urban form and function as already mentioned, urban functions being assumed as distributed by transportation networks \mbox{%DIFAUXCMD
\citep{raimbault2018caracterisation}}%DIFAUXCMD
. In general, the interactions between transportation networks and territories have been shown to be central processes in urban dynamics \mbox{%DIFAUXCMD
\citep{espacegeo2014effets}}%DIFAUXCMD
.
}


\DIFadd{We thus introduce in this paper a multi-dimensional percolation method, and apply }\DIFaddend it to the endogenous characterization of urban regions, \DIFdelbegin \DIFdel{and }\DIFdelend \DIFaddbegin \DIFadd{to finally }\DIFaddend compute stylized sustainability indicators on the constructed regions. \DIFaddbegin \DIFadd{These sustainability measures are extrapolated from gravity flows, and are used as a proof-of-concept of how this work can be applied towards policy-making. As detailed below, we use paradigmatic opposed dimensions of sustainability which are greenhouse gases emissions and economic productivity.
}\DIFaddend 


\DIFdelbegin \DIFdel{Our contribution relies }\DIFdelend \DIFaddbegin \DIFadd{The originality of our contribution relies thus }\DIFaddend on several points: (i) this is to the best of our knowledge the first time a multi-dimensional percolation method is applied to urban systems; (ii) we furthermore apply it on the significant spatial extent of all European Union; and (iii) we link the clusters obtained with simple sustainability measures.
\DIFaddbegin 


\DIFaddend The rest of this paper is organized as follows: we first describe the multi-dimensional percolation heuristic, the data and variables to which it is applied \DIFaddbegin \DIFadd{and how the network is constructed}\DIFaddend , and the indicators used to characterized the sustainability of clusters produced. We then describe the results of applying this method to population and \DIFdelbegin \DIFdel{network variables }\DIFdelend \DIFaddbegin \DIFadd{road network indicators }\DIFaddend for the whole European Union, focusing on the endogenous regions produced and their sustainability properties. We finally discuss possible developments and the implications of this methodology to the design of policies.



%%%%%%%%%%%%%%%%%
\section{Methods}
%%%%%%%%%%%%%%%%%




%%%%%%%%%%%%%%%%%
\subsection{Multi-dimensional percolation}



%%%%%%%%%%%%%%%%%%%%
\begin{figure*}[ht] 
%\resizebox{10cm}{7cm}
  {\includegraphics[width=\linewidth]{figures/principle.pdf}}
  \centering
  \caption{Schematic representation of the multi-dimensional network percolation heuristic. \DIFdelbeginFL \DIFdelFL{We show a stylized configuration with two }\DIFdelendFL \DIFaddbeginFL \DIFaddFL{Two }\DIFaddendFL layers \DIFdelbeginFL \DIFdelFL{having the same nodes}\DIFdelendFL \DIFaddbeginFL \DIFaddFL{are considered here}\DIFaddendFL , \DIFdelbeginFL \DIFdelFL{and the }\DIFdelendFL \DIFaddbeginFL \DIFaddFL{with similar nodes but different }\DIFaddendFL links \DIFdelbeginFL \DIFdelFL{within }\DIFdelendFL \DIFaddbeginFL \DIFaddFL{in }\DIFaddendFL each\DIFaddbeginFL \DIFaddFL{. To each }\DIFaddendFL layer \DIFaddbeginFL \DIFaddFL{is associated a node variable. In each layer, links }\DIFaddendFL are created \DIFdelbeginFL \DIFdelFL{following the }\DIFdelendFL \DIFaddbeginFL \DIFaddFL{according to a distance threshold (}\DIFaddendFL percolation radius\DIFdelbeginFL \DIFdelFL{$r_0$ }\DIFdelendFL \DIFaddbeginFL \DIFaddFL{) }\DIFaddendFL and \DIFdelbeginFL \DIFdelFL{the thresholds for the layer variables}\DIFdelendFL \DIFaddbeginFL \DIFaddFL{a variable value threshold}\DIFaddendFL . The final clusters \DIFaddbeginFL \DIFaddFL{(bottom in color) }\DIFaddendFL are \DIFaddbeginFL \DIFaddFL{obtained by considering }\DIFaddendFL the \DIFdelbeginFL \DIFdelFL{superposition of these. The green points give examples of starting points }\DIFdelendFL \DIFaddbeginFL \DIFaddFL{links present }\DIFaddendFL in \DIFdelbeginFL \DIFdelFL{the case of a propagation heuristic}\DIFdelendFL \DIFaddbeginFL \DIFaddFL{each layer}\DIFaddendFL .\label{fig:method}}
\end{figure*}
%%%%%%%%%%%%%%%%%%%%


\DIFaddbegin \subsubsection{Proposed heuristic}

\DIFaddend Percolation processes in multilayer networks have been proposed as an extension within simple networks \citep{boccaletti2014structure}. A generalization of epidemic spreading can for example be achieved using this framework \citep{son2012percolation}. In the case of multilayer networks sharing the same nodes for all layers, often called multiplex networks \DIFaddbegin \DIFadd{\mbox{%DIFAUXCMD
\citep{aleta2018multilayer}}%DIFAUXCMD
}\DIFaddend , bond percolation has also been studied \citep{hackett2016bond}. \DIFdelbegin %DIFDELCMD < 

%DIFDELCMD < %%%
\DIFdelend In the case of our heuristic, bond percolation is operated between two cells given a distance threshold, and furthermore with a threshold parameter for each layer assuming a node function within each layer. The distance-based connection is similar to generative processes for random euclidian networks \citep{penrose1999k}.

\DIFaddbegin \DIFadd{We do not call our method ``multi-layer percolation'', since nodes are common between layers. We show in Fig.~\ref{fig:method} for a schematic representation of the method. It can be implemented with a propagation heuristic or directly working on adjacency matrices. The rationale behind combining a thresholding for each layer variable with a distance thresholding relies on the idea that a first component for two points to interact is a large enough proximity. The second component is a strong enough intensity of the urban dimension captured by each layer, simultaneously for all layers considered. This recalls Tobler's first law of geography \mbox{%DIFAUXCMD
\citep{tobler2004first} }%DIFAUXCMD
in a multi-dimensional way.
}


\subsubsection{Formal description}

\DIFaddend More formally, let assume a set of nodes $V = v_i$ common to all layers \DIFaddbegin \DIFadd{of the network}\DIFaddend , and layers edges $E_j$ \DIFaddbegin \DIFadd{for layer $j$ }\DIFaddend taken as empty at the initial state of the algorithm. Each node has a value of the considered variables associated to each layer, written \DIFaddbegin \DIFadd{as a matrix }\DIFaddend $v_{ij}$. \DIFaddbegin \DIFadd{The algorithm works as follows:
}\begin{enumerate}
	\item \DIFadd{Percolation is first done within each layer. }\DIFaddend For each layer, a link $e_{kl} \in E_j$ is created if $d(v_k,v_l) < r_0$ where $d$ is the distance between the nodes (which can be any distance) and $r_0$ the percolation radius, and if $v_{kj} > \theta_j$ and $v_{lj} > \theta_j$ where $\theta_j$ is the threshold for layer $j$.
	\DIFdelbegin \DIFdel{The }\DIFdelend \DIFaddbegin \item \DIFadd{Layers are combined, by computing the }\DIFaddend final percolated network edges $E$ \DIFdelbegin \DIFdel{is composed by }\DIFdelend \DIFaddbegin \DIFadd{as the }\DIFaddend links contained within all layers simultaneously. The multi-dimensional percolation clusters are then the connected components of this network $(V,E)$.
\DIFaddbegin \end{enumerate}

 
\DIFaddend The parameters implied in this heuristic are the percolation radius $r_0$ and the percolation thresholds \DIFdelbegin \DIFdel{$\theta_j$ }\DIFdelend \DIFaddbegin \DIFadd{$\vec{\theta} = \theta_j$ }\DIFaddend for each layer $j$\DIFdelbegin \DIFdel{, allowing a flexible application through parametrization. }\DIFdelend \DIFaddbegin \DIFadd{. Varying these parameters allows considering different levels of percolation.
}\DIFaddend 



\DIFdelbegin \DIFdel{Note that we do not call our method ``multi-layer percolation'', as it is not strictly multi-layer since nodes are common.
The term of multi-dimensional percolation is more suited to the use }\DIFdelend \DIFaddbegin \subsubsection{Application to gridded networks}

\DIFadd{The previous generic method must be applied to a consistent urban network, in the sense }\DIFaddend of multiple variables \DIFdelbegin \DIFdel{and thresholds. The method works with an arbitrary number of layers. See Fig.~\ref{fig:method} for a schematic representation of the method. It can be implemented with a propagation heuristic or directly working on adjacency matrices. The rationale behind the conjunction of the thresholding of each layer variable and the distance thresholding relies on the idea that two points will interact is they are close enough, but also if they have a strong enough intensity of the activity or dimension captured by each layer, simultaneously for all layers considered.
This recalls Tobler's first law of geography \mbox{%DIFAUXCMD
\citep{tobler2004first} }%DIFAUXCMD
in a multi-dimensional way.
}\DIFdelend \DIFaddbegin \DIFadd{associated to spatial nodes. To each variable then corresponds a network layer. We propose to work on gridded networks, namely nodes regularly spaced in two dimensions. In practice, these nodes will be the center of raster cells. We will consider two layers, one defined by population density and the other by road network structure indicators computed at the same resolution.
}\DIFaddend 



\DIFdelbegin %DIFDELCMD < \subsection{Empirical data}
%DIFDELCMD < %%%
\DIFdelend \DIFaddbegin \subsection{Empirical data and network construction}
\DIFaddend 


We \DIFdelbegin \DIFdel{apply the heuristic to urban morphology and road network topology measures in Europe. The idea to combine urban form with network topology measures relies on the capture of the link between urban form and function as already mentioned, urban functions being assumed as distributed by transportation networks \mbox{%DIFAUXCMD
\citep{raimbault2018caracterisation}}%DIFAUXCMD
.
}%DIFDELCMD < 

%DIFDELCMD < %%%
\DIFdel{More precisely, a }\DIFdelend \DIFaddbegin \DIFadd{detail now how the empirical layers were computed and the network constructed. A }\DIFaddend grid of population density morphology indicators \DIFdelbegin \DIFdel{and road network topology indicators }\DIFdelend has been computed on spatial moving windows of width 50km for all European Union by \DIFdelbegin \DIFdel{\mbox{%DIFAUXCMD
\cite{raimbault2018urban}}%DIFAUXCMD
}\DIFdelend \DIFaddbegin \DIFadd{\mbox{%DIFAUXCMD
\cite{raimbault2018calibration}}%DIFAUXCMD
}\DIFaddend , with an offset resolution of 5km. \DIFaddbegin \DIFadd{From this study we get the aggregated population, producing a raster grid of population with a resolution of 5km.
}

\DIFadd{Furthermore, road network topology indicators were computed at a similar resolution by \mbox{%DIFAUXCMD
\cite{raimbault2018urban}}%DIFAUXCMD
. In practice, (i) the full OpenStreetMap road network for Europe was simplified at the minimal resolution of 100m, keeping the topological properties and link attributes (including maximal speed for example); (ii) for each cell of the population morphology raster, the road network within the same spatial window of width 50km was retrieved; (iii) network structure indicators (summary statistics, centralities, etc.) were computed on this local network, providing a network indicators raster at the same resolution than population.
}

\DIFaddend We use this data to construct a two layers abstract network: a layer which variable is given by population density, and a second layer which variable is given by a \DIFdelbegin \DIFdel{network variable}\DIFdelend \DIFaddbegin \DIFadd{road network indicator}\DIFaddend . Nodes are the center of cells (thus disposed in space on a grid of step 5km). We test \DIFaddbegin \DIFadd{several possible networks by varying the road indicator taken into account for }\DIFaddend the \DIFdelbegin \DIFdel{variable characterizing the }\DIFdelend second layer\DIFdelbegin \DIFdel{among the following characteristics of the road network within the corresponding window:
}\DIFdelend \DIFaddbegin \DIFadd{. In particular, we test the following indicators:
}\begin{enumerate}
	\item \DIFaddend number of edges $N_E$\DIFdelbegin \DIFdel{, }\DIFdelend \DIFaddbegin \DIFadd{;
	}\item \DIFaddend number of vertices $N_V$\DIFdelbegin \DIFdel{, }\DIFdelend \DIFaddbegin \DIFadd{;
	}\item \DIFaddend cyclomatic number $\mu$ \DIFdelbegin \DIFdel{and }\DIFdelend \DIFaddbegin \DIFadd{which is defined by $\mu = N_E - N_V + c$ where $c$ is the number of connected components of the graph; this measure captures the number of distinct cycles in the graph
	}\item \DIFaddend euclidian efficiency $v_0$\DIFdelbegin \DIFdel{. These measures capture }\DIFdelend \DIFaddbegin \DIFadd{, defined by \mbox{%DIFAUXCMD
\cite{banos2012towards}}%DIFAUXCMD
, as the average rate between network distance and euclidian distance for all couples of links
}\end{enumerate}

\DIFadd{The choice of these measures is aimed at capturing basic aspects of network structure, and }\DIFaddend functional properties especially for the two last. \DIFaddbegin \DIFadd{Indeed, euclidian efficiency measures how the network is performant to link nodes, while the number of cycles is linked to redundancy of paths and in a way to robustness. These choice are arbitrary, but several aspects of transportation networks are still captured by these indicators. An increase of each is related to a more performant network regarding different dimensions, what is relevant for our application. A systematic study with other indicators such as centralities or accessibilities is out of the scope of this paper.
}\DIFaddend 



The percolation on such an abstract network is a necessary condition in our case to link the different dimensions considered, namely population distribution and local road network properties. We have therefore two levels of networks in our approach, namely the physical road network which local properties are taken here as input, and the abstract two layer network on which we do the percolation.
\DIFaddbegin 

\DIFaddend We will in the following write $\theta_P$ for the threshold parameter of the population layer, and $\theta_N$ for the threshold parameter of the network layer. In practice, these parameters will be given in the following as quantile level of the corresponding variable, for an easier interpretation and conception of experience plans. The name of the \DIFdelbegin \DIFdel{variable }\DIFdelend \DIFaddbegin \DIFadd{road network indicator }\DIFaddend considered will be written $v_N$.


%%%%%%%%%%%%%%%%
\subsection{Sustainability indicators}


As already detailed, \DIFdelbegin \DIFdel{recent forms of urbanization, in particular integrated mega-city regions , may imply different patterns of economic and transportation flows and thus exhibit various performances regarding different indicators }\DIFdelend \DIFaddbegin \DIFadd{urban regions may perform more or less well regarding different dimensions }\DIFaddend of sustainability. We propose to use the endogenous definition of regional urban systems produced by the percolation algorithm to evaluate their sustainability, in terms of conflicting objectives of economic integration and greenhouse gases emissions. The definition of sustainability, or sustainable development, is by essence multi-dimensional \citep{viguie2012trade}. Its characterization as quantitative indicators is even more subject to numerous degrees of freedom. We work here with \DIFaddbegin \DIFadd{these }\DIFaddend two stylized indicators for two conflicting dimensions, as a proof-of-concept. \DIFaddbegin \DIFadd{These dimensions can furthermore be approximated indirectly from gravity models as we will describe below. By introducing other datasets, our work could be extended to more realistic indicators and other dimensions.
}\DIFaddend 

We use the EDGAR database \citep{janssens2017edgar} (version 4.3.2) for local grid estimates of greenhouse gases emissions. We use the latest year available, namely 2012. As its resolution is much smaller than our indicator grid, we aggregate the emissions on the closer indicator point for each cell of the emission database. Since according to \cite{lashof1990relative} most of the greenhouse effect is caused by $\textrm{CO}_2$, and as in terms of emissions in the database we find that it represents $98.2\%$ in mass proportion of all gases, we only consider it.

Applying a gravity model to each region, we estimate abstract transportation flows within each \DIFdelbegin \DIFdel{and use these to extrapolate emissions from the actual local emission from the Edgar database, and economic activities with a scaling law of population}\DIFdelend \DIFaddbegin \DIFadd{with a gravity model}\DIFaddend . More precisely, following \cite{raimbault2018indirect}, a potential flow between two points $i$ and $j$ can be estimated with the following expression 

\begin{equation}
\phi_{ij}^{(k)} = \left(\frac{v^{(k)}_i v^{(k)}_j}{(\sum_l v_l)^2}\right)^\gamma \cdot \exp\left(\frac{-d_{ij}}{d_0}\right)
\end{equation}

where $v^{(k)}_i$ are either population or effective local GHG emissions \DIFaddbegin \DIFadd{computed with the EDGAR database }\DIFaddend (indexed by $k = 1,2$ respectively), $d_{ij}$ the distance between the two points, $d_0$ a distance decay parameter, and $\gamma$ a scaling parameter. Indeed, the economic activity follows relatively well scaling laws of populations \citep{bettencourt2007growth}, the exponent being dependant on the activity and the definition of areas on which it is estimated~\citep{cottineau2017diverse}. \DIFdelbegin %DIFDELCMD < 

%DIFDELCMD < %%%
\DIFdel{The }\DIFdelend \DIFaddbegin \DIFadd{The distance decay captures the geographical span of potential interactions. These two parameters $\gamma,d_0$ are left free and varying them allows considering stylized configurations, such as long or short span interactions, or infra- or supra-linear scaling activities. Finally, considering the flow with the population variable ($k=1$) provides a proxy for economic flows, while with the GHG emissions ($k=2$) it provides a proxy for emissions in relation with this economic activity. Indeed, effective emissions are linked to local emissions and transportation emissions linked to the intensity of flows.
}

\DIFadd{We then consider the }\DIFaddend sum of all \DIFaddbegin \DIFadd{these }\DIFaddend flows of points within the geographical span of \DIFdelbegin \DIFdel{the cluster (that we approximate }\DIFdelend \DIFaddbegin \DIFadd{a given cluster of nodes in our network. These clusters are obtained with the percolation method described above, and are numbered by $1 \leq c \leq C$. For the sake of simplicity, we approximate the corresponding geographical area }\DIFaddend as the convex Hull envelope of \DIFdelbegin \DIFdel{its points ), allows us to approximate the cumulated potential emissions and economic activity. Writing clusters }\DIFdelend \DIFaddbegin \DIFadd{the points in the cluster, that we write }\DIFaddend $K_c$\DIFdelbegin \DIFdel{as this set of points, we }\DIFdelend \DIFaddbegin \DIFadd{. By cumulating the flows, we therefore }\DIFaddend define the total economic flow \DIFdelbegin \DIFdel{by $E_c = \sum_{i,j \in K_c} \phi_{ij}^{(1)}$ }\DIFdelend \DIFaddbegin \DIFadd{as the sum of economic flows by 
}

\begin{equation}
\DIFadd{E_c = \sum_{i,j \in K_c} \phi_{ij}^{(1)}
}\end{equation}

\DIFaddend and the total emissions due to flows by
\DIFdelbegin \DIFdel{$G_c = \sum_{i,j \in C_c} \phi_{ij}^{(2)}$}\DIFdelend \DIFaddbegin 

\begin{equation}
\DIFadd{G_c = \sum_{i,j \in C_c} \phi_{ij}^{(2)}
}\end{equation}

\DIFadd{These indicators have no interpretable unit and need to be renormalized}\DIFaddend . This allows \DIFdelbegin \DIFdel{to define }\DIFdelend \DIFaddbegin \DIFadd{defining }\DIFaddend a relative economic inefficiency by 
\DIFdelbegin \DIFdel{$e_c = 1 - \frac{\max_c E_c - E_c}{\max_c E_c - \min_c E_c}$ and }\DIFdelend \DIFaddbegin 

\begin{equation}
\DIFadd{e_c = 1 - \frac{\max_c E_c - E_c}{\max_c E_c - \min_c E_c}
}\end{equation}

\DIFadd{where $\max_c E_c$ (resp. $\min_c E_c$) is the maximal (resp. minimal) value of $E_c$ across all clusters. We define in a similar way the }\DIFaddend relative potential emissions by
\DIFdelbegin \DIFdel{$g_c = \frac{\max_c G_c - G_c}{\max_c G_c - \min_c G_c}$. }\DIFdelend \DIFaddbegin 

\begin{equation}
\DIFadd{g_c = \frac{\max_c G_c - G_c}{\max_c G_c - \min_c G_c}
}\end{equation}

\DIFadd{where $\max_c G_c$ (resp. $\min_c G_c$) is the maximal (resp. minimal) value of $G_c$ across all clusters.
}

\DIFaddend Both indicators should be minimized for sustainability. \DIFdelbegin \DIFdel{Normalized }\DIFdelend \DIFaddbegin \DIFadd{Their value is dependant on the number of clusters and their extent, i.e. the geographical surface they cover. To be able to compare values across different clusterings (corresponding to different parameter values for the percolation heuristic), we finally define normalized }\DIFaddend indicators $\tilde{e}_c,\tilde{g}_c$ \DIFdelbegin \DIFdel{are defined }\DIFdelend in a similar way, but the extrema being computed on all other possible urban configurations with the same $\gamma,d_0$ values.

Using these potential flows follows the logic of \cite{arbabi2019development} which shows a need for improved intra-city-region mobility in England and Wales. Considering the regions as entities in which such transportation development policies can more easily been developed, we look at the sustainability of different possible regions if these potential flows were realized. Varying the parameters $\gamma$ and $d_0$ allows to control for the economic activity considered (high $\gamma$ values correspond to high added-value activities) and the span of interactions through $d_0$.

\DIFaddbegin \DIFadd{For descriptive purposes, we also consider summary measures of clusters, as the population $P_c$ and effective emissions $EM_c$ taken as the sum of population (resp. emissions) of the points in $K_c$.
}


\DIFaddend \section{Results}

\subsection{Implementation}


In practice, the analysis is implemented using R and the igraph package. Source code, data and results are available on the open git repository of the project at \url{https://github.com/JusteRaimbault/UrbanMorphology}. The network is constructed by superposing the population density layer with the network layer, starting from the 5km resolution spatial fields for morphological and network indicators. This network is filtered with the threshold parameters for each layer and with the radius parameter. Connected components yield the clusters that we interpret as endogenous regions.



\DIFdelbegin \DIFdel{We recall that the euclidian performance of the network \mbox{%DIFAUXCMD
\citep{banos2012towards} }%DIFAUXCMD
is in our case $<d_e/d_n>$ where the average is taken on all origin-destination pairs in the network, $d_e$ is the euclidian distance and $d_n$ the network distance. Thus, it indeed increases with network performance, in consistence with the use done here through thresholding.
}%DIFDELCMD < 

%DIFDELCMD < %%%
\DIFdelend %%%%%%%%%%%%%%%%%%%%
\begin{figure*}[ht] 
%\resizebox{10cm}{7cm}
  {\includegraphics[width=0.49\linewidth]{figures/abssize_nodes.png}}
  {\includegraphics[width=0.49\linewidth]{figures/relsize_nodes.png}}
  \centering
  \caption{Percolation transition. On the left, we plot the size of the largest cluster in each configuration in terms of nodes, as a function of the percolation radius $r_0$. Color gives the other percolation parameters. On the right, the plot is similar but with the size relative to the size of the largest cluster obtained with the maximal radius in each configuration.\label{fig:percolation}}
\end{figure*}
%%%%%%%%%%%%%%%%%%%%

%%%%%%%%%%%%%%%%%%%%
\begin{figure*}[ht] 
%\resizebox{10cm}{7cm}
  {\includegraphics[width=\linewidth]{figures/fractaldimension.png}}
  \centering
  \caption{Fractal dimension. We plot for each parametrization given by the curve color the evolution of the fractal dimension $\alpha$ as a function of $r_0$. Standard errors are not plotted for readability.\label{fig:fractaldim}}
\end{figure*}
%%%%%%%%%%%%%%%%%%%%

%%%%%%%%%%%%%%%%%%%
\subsection{Percolation transition and fractal dimension}



In its application to road networks by \cite{arcaute2016cities}, the structure of the national urban system for UK is captured by studying the percolation transition, i.e. the variation of the size of the largest cluster as a function of the percolation radius. As this signature is tightly linked to historical, cultural and geographical conditions, the application to different urban systems should yield different results. We study here this property, for different threshold parameter values. We make the radius vary betweem 8km and 100km with a one km step, have a fixed population threshold $\theta_P = 0.85$, test all \DIFdelbegin \DIFdel{network variables}\DIFdelend \DIFaddbegin \DIFadd{road network indicators}\DIFaddend , and three network \DIFaddbegin \DIFadd{layer }\DIFaddend thresholds $\theta_N \in \{ 0 ; 0.8 ; 0.95 \}$.

The absolute and relative sizes of the largest cluster are plotted in Fig.~\ref{fig:percolation} as a function of the percolation radius. This aspect first gives methodological information on multilayer percolation. Indeed, comparing the result with $\theta_N = 0$ (single layer percolation) with positive values of $\theta_N$ shows a significantly different behavior. As expected, absolute size are much smaller, but when looking at relative sizes we observe that the abrupt steps typical to percolation transitions have different distributions across the different parametrizations. \DIFaddbegin \DIFadd{This result is particularly interesting regarding the first axis of our research question, as it shows that the structure of clusters obtained is not only due to the population layer, and that the multi-dimensional percolation captures a complementary signal.
 }

\DIFaddend The more regular curve seems to be the standard percolation on population only, whereas at $\theta_N = 0.95$, different \DIFdelbegin \DIFdel{network variables }\DIFdelend \DIFaddbegin \DIFadd{road network indicators }\DIFaddend produce either very early transitions (for $\mu$ for example) or very late (for $N_V$). Also, changing of scale compared to \cite{arcaute2016cities} gives more steps and less abrupts curves in general, confirming the integration of subsystems with different structures in our analysis and the importance of scale in such analysis. As the addition of a layer also changes drastically the results, one should stay careful when switching from a mono-dimensional percolation to a multi-dimensional percolation.



% ! largest cluster size in terms of edges makes no sense with the implementation we took - nodes only
%      nwcol         popthq         nwthq          radius         nwparam             fractdim     
% ecount  :186   Min.   :0.85   Min.   :0.00   Min.   :  8000   Length:1023        Min.   :0.9239  
% euclPerf:279   1st Qu.:0.85   1st Qu.:0.00   1st Qu.: 31000   Class :character   1st Qu.:1.2639  
% mu      :279   Median :0.85   Median :0.80   Median : 54000   Mode  :character   Median :1.3306  
% vcount  :279   Mean   :0.85   Mean   :0.55   Mean   : 54000                      Mean   :1.3225  
 %               3rd Qu.:0.85   3rd Qu.:0.95   3rd Qu.: 77000                      3rd Qu.:1.3909  
 %               Max.   :0.85   Max.   :0.95   Max.   :100000                      Max.   :1.6058                                                                                    NA's   :337     
%   fractdimsd     fractdimadjrsquared   fractrelsd    
% Min.   :0.0544   Min.   :0.5111      Min.   :0.0395  
% 1st Qu.:0.0966   1st Qu.:0.7016      1st Qu.:0.0700  
% Median :0.1328   Median :0.7467      Median :0.1043  
% Mean   :0.1317   Mean   :0.7456      Mean   :0.1009  
% 3rd Qu.:0.1578   3rd Qu.:0.7981      3rd Qu.:0.1232  
% Max.   :0.2381   Max.   :0.8642      Max.   :0.1958  
% NA's   :337      NA's   :337         NA's   :337  




We study also the evolution of the fractal dimension of clusters as a function of $r_0$\DIFaddbegin \DIFadd{, to verify how the initial percolation approach is robust to multi-dimensionality}\DIFaddend . Following \cite{arcaute2016cities}, we estimate the fractal dimension \DIFdelbegin \DIFdel{of clusters $\alpha$ }\DIFdelend \DIFaddbegin \DIFadd{$\alpha$ of clusters }\DIFaddend with a simple OLS regression between cluster size and cluster diameter, namely 
\DIFdelbegin \DIFdel{$\log N_c = k + \alpha \cdot \log \delta_c$ }\DIFdelend \DIFaddbegin 

\begin{equation}
\DIFadd{\log N_c = k + \alpha \cdot \log \delta_c
}\end{equation}

\DIFaddend where $N_c$ is the size of cluster $c$ and $\delta_c$ its diameter. \DIFaddbegin \DIFadd{These estimations are shown in Fig.~\ref{fig:fractaldim}. }\DIFaddend As a negative result, \DIFdelbegin \DIFdel{that }\DIFdelend \DIFaddbegin \DIFadd{which }\DIFaddend could be due to the abstract nature of our network, a clear maximum in the value of the fractal dimension can not be found. Either it is located at \DIFaddbegin \DIFadd{a }\DIFaddend resolution that our method can not \DIFdelbegin \DIFdel{reached }\DIFdelend \DIFaddbegin \DIFadd{reach }\DIFaddend due to the minimal 5km limit imposed by the abstraction \DIFaddbegin \DIFadd{in the network construction}\DIFaddend , or it does not exist when coupling dimensions. Determining which assumption is more plausible is out of the scope of this paper.
\DIFaddbegin 

\DIFaddend We do not plot the standard error $\sigma$ of fractal dimensions \DIFaddbegin \DIFadd{(obtained as the estimation error in the OLS) }\DIFaddend for visibility purposes, but their relative value given by $\alpha / \sigma\left[\alpha\right]$ is in average 0.10 and in maximum 0.196 on all points, meaning that these estimations remain however consistent.
\DIFaddbegin 

\DIFaddend Regarding the variability \DIFaddbegin \DIFadd{of fractal dimension }\DIFaddend as a function of the percolation radius $r_0$, \DIFdelbegin \DIFdel{studying the }\DIFdelend \DIFaddbegin \DIFadd{we study the possible existence of a significant maximum when $r_0$ varies. We therefore simply consider the }\DIFaddend difference $(\alpha - \sigma\left[\alpha\right])_M - (\alpha - \sigma\left[\alpha\right])_m$ where the first is taken at \DIFdelbegin \DIFdel{maximum for }\DIFdelend \DIFaddbegin \DIFadd{the maximum value of }\DIFaddend $\alpha$ and the other at \DIFdelbegin \DIFdel{minimum, shows }\DIFdelend \DIFaddbegin \DIFadd{its minimum value. This intuitively corresponds to checking if confidence intervals do not overlap between the maximum and the minimum of the curve. We find }\DIFaddend that the configuration for $\mu$ and $\theta_N=0.95$ has a clearly significant maximum (difference at 0.38). For this coupling, the endogenous structure given by the maximum may be defined. Other configurations yield non-significant maximums \DIFaddbegin \DIFadd{(negative values of the difference)}\DIFaddend .


\DIFaddbegin \DIFadd{This study of percolation transition and fractal dimension thus shows that our multi-dimensional percolation heuristic remains relevant, as results analog but qualitatively different to the one-dimensional approach can be obtained.
}


\DIFaddend %  fractdimsignif
%# A tibble: 11 x 5
%# Groups:   nwcol, popthq, nwthq [?]
%   nwcol    popthq nwthq nwparam       signif
%   <fct>     <dbl> <dbl> <chr>          <dbl>
% 1 ecount     0.85  0    ecount0      -0.105 
% 2 ecount     0.85  0.8  ecount0.8    -0.0452
% 3 euclPerf   0.85  0    euclPerf0    -0.108 
% 4 euclPerf   0.85  0.8  euclPerf0.8   0.0360
% 5 euclPerf   0.85  0.95 euclPerf0.95 -0.108 
% 6 mu         0.85  0    mu0          -0.108 
% 7 mu         0.85  0.8  mu0.8        -0.108 
% 8 mu         0.85  0.95 mu0.95        0.375 
% 9 vcount     0.85  0    vcount0      -0.105 
%10 vcount     0.85  0.8  vcount0.8    -0.0416
%11 vcount     0.85  0.95 vcount0.95   -0.0868


%%%%%%%%%%%%%%%%%%%%
\begin{figure}[h!] 
%\resizebox{10cm}{7cm}
{\includegraphics[width=0.49\textwidth]{figures/totalPop4183694_00056402_ecount850_radius8000.png}}
  {\includegraphics[width=0.49\linewidth]{figures/totalPop2219780_36719597_vcount378_radius8000.png}}\\
  {\includegraphics[width=0.49\linewidth]{figures/totalPop1474347_36891685_vcount595_radius50000.png}}
   {\includegraphics[width=0.49\linewidth]{figures/totalPop1474347_36891685_euclPerf0_00507404608502498_radius54000.png}}
  \centering

  \caption{Examples of obtained clusters for different parameter values. In the top-right case for example ($\theta_P = 0.9$, $\theta_N = 0.8$, variable \texttt{vcount},$r_0 = 8km$), we obtain the urban regions of West midlands and London in the UK, Randstad merged with Rhein-Rhur and Rhein-Main in Germany, Paris in France, also with capital cities such as Copenhaguen, Stockholm and Helsinki. There is no cluster in South Europe in that case, due to the high population density threshold.\label{fig:exclusters}}
\end{figure}
%%%%%%%%%%%%%%%%%%%%


%%%%%%%%%%%%%%%%%%%%
\begin{figure}[h!] 
%\resizebox{10cm}{7cm}
  {\includegraphics[width=0.7\linewidth]{figures/full_effective_pareto.png}}
  \centering
  \caption{Point clouds of region-level indicators, namely population and emissions, for different parametrizations, given by the color. Each point represent an endogenous urban region.\label{fig:paretos}}
\end{figure}
%%%%%%%%%%%%%%%%%%%%

%%%%%%%%%%%%%%%%%%%%
\subsection{Extracting endogenous mega-city regions}


We now switch the experience plan to a full grid, for parameters $r_0$, $\theta_P$, $\theta_N$ and the \DIFdelbegin \DIFdel{network variable }\DIFdelend \DIFaddbegin \DIFadd{road network indicator }\DIFaddend considered, and also make $\gamma$ and $d_0$ vary. We systematically explore the clusters obtained for 4800 parameter configurations, such that for all \DIFdelbegin \DIFdel{network variables}\DIFdelend \DIFaddbegin \DIFadd{road network indicator}\DIFaddend , $\theta_P \in \{ 0.8 ; 0.9 ; 0.95 \}$, $\theta_N \in \{0 ; 0.8 ; 0.95 \}$, $r_0 \in \{ 8 ; 10 ; 15 ; 20 ; 50\}$ km, $\gamma \in \{ 0.5 ; 1 ; 1.5 ; 2\}$, and $d_0 \in \{ 0.1 ; 1 ; 10 ; 50 ; 100\}$ km.


We obtain very different endogenous morphologies for the different parametrizations. Maps reveal that some configurations resemble the actual distribution of European mega-city regions, which are functionally integrated polycentric urban areas \citep{hall2006polycentric}. These are here defined endogenously from the bottom-up and have a priori no reason to coincide with these functional regions. We show some examples in Fig.~\ref{fig:exclusters}. The first map of this figure, obtained for high population and network thresholds ($\theta_P = 0.95$ and $\theta_N = 0.9$), but a low radius $r_0 = 8$km and edge count $N_E$ \DIFdelbegin \DIFdel{as network variable}\DIFdelend \DIFaddbegin \DIFadd{to define the road network layer}\DIFaddend , include several mega-city regions described by \citep{hall2006polycentric}, namely London metropolitan area, the Randstad in Netherland, the Rhein-Main and Rhein-Ruhr in Germany, Greater Paris in France, Brussels area in Belgium. The same parameters with $\theta_N = 0$ yield not exactly the same regions, as confirmed by the transition curves in Fig.~\ref{fig:percolation}, what means that our approach taking into account two dimensions may capture effective processes of mega-city regions, in particular by including the road network which is crucial as these regions are integrated in terms of flows. The bottom-left map show an example of large clusters emerging in UK and in the center of Europe, the South remaining largely disconnected. Finally, the last map shows the result obtained with a very high radius $r_0 = 54$km, with a giant cluster spanning most of Europe. UK is still disconnected and the transition where it connects happens at $r_0 = 55$km. This does not necessarily mean that UK should be disconnected from continental Europe, as we considered geographic distances only, hiding the high speed connection of the Channel tunnel.


The behavior of sustainability indicators for different population, network and distance thresholds yield different distributions of performances across clusters within a configuration but also between configurations. Before considering the flow-based indicators described above, we can already study basic \DIFaddbegin \DIFadd{summary }\DIFaddend measures such as population $P_c$ and effective emissions $EM_c$\DIFdelbegin \DIFdel{, taken as the sum within the cluster of their values at each point}\DIFdelend . We show in Fig.~\ref{fig:paretos} point clouds of $\log EM_c$ against $1 - \log P_c$ for some configurations. Indeed, regarding the population it contains, an area can be more or less efficient in terms of emissions. Seeing the population as an objective to be maximized (thus the plotted value to be minimized) \DIFaddbegin \DIFadd{while the emissions must be minimized}\DIFaddend , we observe a Pareto front for all points (i.e. all clusters across all configurations)\DIFdelbegin \DIFdel{, but also }\DIFdelend \DIFaddbegin \DIFadd{. Given different dimensions to be minimized, a Pareto front consists of the points which are not Pareto-dominated by any other point, i.e. that there exists no other point performing best on all objectives. In practice, this yields optimization compromises in the context of multi-objective optimization, when no aggregation of the dimensions is possible or desirable. In Fig.~\ref{fig:paretos}, the front is the lower bound of the point cloud. We also find }\DIFaddend no dominating point for each configuration\DIFdelbegin \DIFdel{. }\DIFdelend \DIFaddbegin \DIFadd{, i.e. that considering point clouds of a single color, a Pareto front with more than one point is still present. }\DIFaddend Some clusters are therefore optimal compromises in the Pareto sense in each configuration, while some are dominated and thus not efficient. \DIFaddbegin \DIFadd{This confirms that urban systems are generally compromises between multiple objectives.
}\DIFaddend 


%DIF <  not really interesting + messy figure -> do not show this, morphology is more interesting
%DIF < \includegraphics[height=0.8\textheight]{figures/aggreg_paretos_radiuspopthq.png}
\DIFdelbegin %DIFDELCMD < 

%DIFDELCMD < %%%
\DIFdelend %%%%%%%%%%%%%%%%%%%%
\begin{figure*}[!ht] 
%\resizebox{10cm}{7cm}
  {\includegraphics[width=\linewidth]{figures/aggreg_morpho_pc1-emissions_targeted.png}}
  \centering
  \caption{Aggregated values of normalized potential emissions $\sum_c \tilde{g}_c$, as a function of the first morphological principal component (PC1), for varying values of parameters $d_G$ (rows) and $\gamma_G$ (columns). Other intermediate values for these parameters yield similar behaviors. As PC1 is mainly linked to monocentricity, there seems to exist an optimal intermediate level of monocentricity for emissions alone. Color level give the share of population within the considered clusters in comparison to all European population.\label{fig:emissions-pc1}}
\end{figure*}
%%%%%%%%%%%%%%%%%%%%


%%%%%%%%%%%%%%%%%%%%
\begin{figure*}[!ht] 
%\resizebox{10cm}{7cm}
  {\includegraphics[width=\linewidth]{figures/aggreg_morpho_relemissions-relefficiency_colpc1_logscale_targeted.png}}
  \centering  
  \caption{Relative potential emissions $\sum_c g_c$ against relative potential economic unefficiency $\sum_c e_c$ (both indicators should be minimized), for varying values of $\gamma_G$ (columns) and $d_G$ (rows). Color level gives the value of PC1, whereas point size gives the share of total population contained within considered areas.\label{fig:paretos-relative}}
\end{figure*}
%%%%%%%%%%%%%%%%%%%%


%%%%%%%%%%%%%
\subsection{Linking urban morphology and sustainability}



%# Cumulative Proportion  0.7296 0.9650 0.99761 1.00000
%                PC1       PC2         PC3         PC4
%moran   -0.3088585 0.9493848 -0.04444327  0.03605266
%avgdist  0.5417362 0.1415668 -0.82239570 -0.10072759
%entropy  0.5108424 0.2140447  0.45847647 -0.69499942
%slope    0.5917502 0.1811415  0.33390034  0.71100630

We now consider the \DIFdelbegin \DIFdel{sustainibility }\DIFdelend \DIFaddbegin \DIFadd{sustainability }\DIFaddend indicators, aggregated for a configuration on all clusters. \DIFaddbegin \DIFadd{An important question is how these relate with measures of urban form \mbox{%DIFAUXCMD
\citep{le2012urban}}%DIFAUXCMD
. }\DIFaddend For a given parametrization of endogenous city regions, one can relate them to morphological indicators for population density spatial distribution\DIFdelbegin \DIFdel{, }\DIFdelend \DIFaddbegin \DIFadd{. Such indicators were }\DIFaddend computed by~\cite{raimbault2018calibration}, \DIFdelbegin \DIFdel{that we average }\DIFdelend \DIFaddbegin \DIFadd{and capture different dimensions of the spatial distribution of population, such as spatial autocorrelation (Moran index), homogeneity (entropy index), hierarchy (rank-size exponent), or level of aggregation (average distance between individuals). We average them here }\DIFaddend on clusters. This establishes a link between urban morphology and sustainibility. A principal component analysis on considered points yield 96\% of variance with two components, and 73\% explained by the first component alone. The first component relates to a level of monocentricity ($PC1 = -0.3\cdot I + 0.54 \cdot \bar{d} + 0.51\cdot \varepsilon + 0.59 \cdot h$ where $I$ is Moran index, $\bar{d}$ average distance, $\epsilon$ entropy, and $h$ level of hierarchy). \DIFaddbegin \DIFadd{In a nutshell, the principal dimension of urban form in differentiating our urban clusters is the level of monocentricity. We can relate it with indicators for emissions and economic efficiency.
}\DIFaddend 


We show in Fig.~\ref{fig:emissions-pc1} the value of $\sum_c \tilde{g}_c$ as a function of the first morphological principal component, for extreme values of $\gamma$ and $d_0$. There seems to exist an optimal intermediate value for PC1 regarding the minimization of normalized indicator for emissions only. This would correspond to an intermediate level of monocentricity, meaning that urban areas which are too polycentric and spread would emit more, but also areas that are too much monocentric. This behavior does not occur for long-range $d_0 = 100$km and low-hierarchy $\gamma=0.5$ interactions. The mostly monocentric but emitting configurations capture most of population (given by the level of color), whereas the intermediate configuration capture around half of the population, what means that these low-emissions potential urban regions can cover a significant part of European population.


However, when considering both emissions and economic indicators, urban form then acts as a compromise variable. We show in Fig.~\ref{fig:paretos-relative} the point clouds of $\sum_c g_c$ against $\sum_c e_c$, which produce clear Pareto fronts, which shape varies with $\gamma$ and $d_0$. As the color level gives the value of PC1, we can see the points on the different fronts with very different morphological properties. In some \DIFdelbegin \DIFdel{case}\DIFdelend \DIFaddbegin \DIFadd{cases}\DIFaddend , highly monocentric areas (yellow points) can be a good compromise, whereas the intermediate optimal for emissions shown before may yield highly inefficient areas (dominated green points). For example, considering the fronts for $\gamma = 2$ which have both very similar shape, the points with the lowest emissions are on the top-left of the front and correspond to the optimal unveiled in Fig.~\ref{fig:emissions-pc1}. These have however a very low economic efficiency (high inefficiency) and small improvements can be done with the points below, before switching to a totally different urban form with a high value of PC1 (yellow points, highly monocentric). Increasing more the economic efficiency is then at the price of much more emissions, with more polycentric areas. This analysis therefore unveils morphological trade-offs, confirming that there is no optimal urban form, but different compromises regarding the conflicting sustainability indicators.



%%%%%%%%%%%%%%%%%
\section{Discussion}


\DIFaddbegin \subsection{From multi-dimensional percolation to the sustainability of mega-city regions}

\DIFadd{From the methodological point of view, we showed that network percolation can successfully be applied to multidimensional urban networks. This requires a consistent overlay within the same nodes of the different dimensions considered. The existence of percolation transitions which are different to the unidimensional case confirms that the approach captures complementary information, and that it could be applied to characterize urban systems in a more refined way. The non-existence of a significant maximum for the fractal dimension remains to be investigated, since it could be due a bias of our abstract network construction. Studies on other dimensions or on non-abstract networks remain to be done to understand how multi-dimensional percolation differs from the unidimensional one.
}


\DIFadd{Regarding the second axis of our research question, we showed that multi-dimensional percolation is a useful tool to extract endogenous mega-urban regions while taking into account complementary aspects of population distribution and performance of transportation networks. Varying the parameters of the percolation algorithm provides a comparative view on possible clustering structures for the European urban system, and corresponding performance in terms of stylized sustainability indicators. Indeed, this work is exploratory in terms of possible definitions of urban subsystems. The fact that some systems obtained coincide with effective functional regions \mbox{%DIFAUXCMD
\citep{hall2006polycentric} }%DIFAUXCMD
shows that some thresholds of population and road network performance intensity capture actual functional linkages. This correspondance could not be predicted a priori nor explained through simple arguments as our approach reconstructs clusters from the bottom-up. Finally, the links between urban form and sustainability indicators made in the last section are also interesting for the management of urban systems, suggesting a certain performance of polycentric systems in particular regarding emissions.
}


\DIFaddend %%%%%%%%%%%%%%%%%
\subsection{Developments}

Further work may consist in the use of calibration heuristics to find in a more robust way optimal parameter values. The OpenMOLE model exploration platform provides a transparent access to genetic algorithms for multi-objective optimization \citep{reuillon2013openmole}. The use of such calibration algorithms would allow to unveil the effective form of Pareto fronts, that we may have missed here through the grid sampling.


An other development would consist in extrapolating transportation flows with a spatially explicit gravity and transportation flow model as a kind of simplified four step model \citep{mcnally2000four}. It could then be adjusted on actual transportation flows emissions database which are also available in the Edgar database. The corresponding gravity parameters could then be used within the economic and emissions potentials, and the sustainability patterns produced compared with the hypothetical ones we produced here.

Finally, an important development would imply crossing our endogenous definitions of urban regions with socio-economic databases, and compute indicators implied in other dimensions of sustainability, for example related to socio-economic inequalities, spatial distribution of accessibilities, or activities with different scaling exponents. This includes the mitigation of spatial inequalities and segregation \citep{tammaru2015multi}, which are an important dimension of \DIFdelbegin \DIFdel{sustainibility}\DIFdelend \DIFaddbegin \DIFadd{sustainability}\DIFaddend .



%%%%%%%%%%%%%%%%%
\subsection{Towards policy applications}


Our work suggests the possibility to design policies in terms of regional integration to increase the sustainability of mega-city regions. The way such results could actually be transferred to policy-making recommandations remains an open question, but Pareto-optimal configurations can be used for the planning of regional transportation networks for example, or to design policies for the distribution of subsidies. Indeed, privileging some infrastructure developments but also collaborations between urban centers can be seen as an aspect of a small scale planning, or territorial strategy. As we integrated potential flows that would result from such development, and consider their economic and emissions consequences, and did it in an endogenous way, we suggest that evidence-based strategies for territorial development at the European level could be inspired by this work. This would naturally imply a more thorough data integration, model calibration and operationalization.

  

%%%%%%%%%%%%%%%%%
\section{Conclusion}


In conclusion, our multilayer percolation approach captures in a way the multi-dimensionality of urban systems and a link between form and function in urban system. \DIFaddbegin \DIFadd{In particular, in our application on the bilayer case of an abstract network constructed from population density and road network indicators, it is shown to capture a different structure than in the unidimensional case. }\DIFaddend Its application to the issue of sustainable mega-city regions \DIFdelbegin \DIFdel{shows its potentialities}\DIFdelend \DIFaddbegin \DIFadd{show how its properties of constructing urban clusters from the bottom-up can be used to study sustainability issues}\DIFaddend . This work also illustrates the importance of following data-driven paradigms even when developing, as what is understood of the behavior of the heuristic is through its application to real data and issues.




\bibliographystyle{jimis-en}
\bibliography{biblio}



%\appendix\footnotesize

%\section{Appendix 1: supplementary sensitivity analyses}



%\sframe{Results: effective emissions}{

%\textit{Effective emissions exhibit a supralinear scaling of population}
%\includegraphics[height=0.83\textheight]{figures/aggreg_effective.png}

%\sframe{Results: all clusters Pareto fronts}{

%\textit{Variation of Pareto front patterns when potential parameter $\gamma,d_0$ vary.}

%\includegraphics[width=0.9\textwidth]{figures/full_paretos.png}


%\sframe{Results: an optimal morphology}{

%\textit{More monocentric areas are more optimal in terms of relative emissions and efficiency ?}

%\includegraphics[width=0.49\textwidth]{figures/aggreg_morpho_pc1-relefficiency.png}
%\includegraphics[width=0.49\textwidth]{figures/aggreg_morpho_pc1-relemissions.png}


%DIF <  TODO : curves for emissions rescale to the same -> link with Caruso profiles ? ; investigate that, and why not for efficiency.
%DIF >   curves for emissions rescale to the same -> link with Caruso profiles ? ; investigate that, and why not for efficiency.





%\section{Acknowledgment}
%

%\section{Biography}
%Here, feel free to add short biographies of the authors.






\end{document}





%%%%%%%%
%% -- TEMPLATES

%\begin{table}
%  \newcolumntype{+}{>{\global\let\currentrowstyle\relax}}
%  \newcolumntype{^}{>{\currentrowstyle}}
%  \newcommand{\rowstyle}[1]{\gdef\currentrowstyle{#1}%
%    #1\ignorespaces
%  }
%  \centering
%  \begin{tabular}{+>{\bfseries}l^c^c^c^c}
%    \hline
%    \rowstyle{\bfseries}
%    & Sepal.Length & Sepal.Width & Petal.Length & Petal.Width\\
%    Setosa & 5.006 & 3.428 & 1.462 & 0.246\\
%    Versicolor & 5.936 & 2.77  & 4.26  & 1.326\\
%    Verginica & 6.588 & 2.974 & 5.552 & 2.026\\
%    \hline
%  \end{tabular}
%  \caption{Morbi malesuada diam at magna condimentum.}
%  \label{tab:example}
%\end{table}

%
%\begin{figure*}[ht] 
%\resizebox{10cm}{7cm}
%  {\includegraphics[width=11cm]{spider.jpg}}
%  \centering
%  \label{frog}
%  \caption{A spider, Picture from Didier Josselin.}
%  \end{figure*}
%  


