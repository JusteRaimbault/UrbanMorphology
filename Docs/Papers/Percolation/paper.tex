\documentclass{jimis-en}
\pdfoutput=1 %forces arXiv/HAL to compile with PDFLaTeX

% Packages
\usepackage{array}
\usepackage{pgfplots}
\usepackage{graphicx}

% header
\title{Identifying European Sustainable Mega-city Regions with Multi-dimensional Network Percolation}
% Endogenous urban regions as an optimal compromise between economic integration and greenhouse gases emissions
% A multi-dimensional percolation approach to characterize sustainable mega-city regions



\author[1,2,3,*]{Juste RAIMBAULT}
\affil[1]{UPS CNRS 3611 ISC-PIF, France} 
\affil[2]{CASA, UCL, UK}
\affil[3]{UMR CNRS 8504 G{\'e}ographie-cit{\'e}s}
 
% corresponding author with his/her email
\corrauthor{juste.raimbault@polytechnique.edu}

%% Information provided by JIMIS:
%ISSN
% DOI 
\doi{10.18713/JIMIS-ddmmyy-v-a}{}
% reviewing process: first date = submission, second date = acceptance 
\review{Month-in-letters Day Year}{Month-in-letters Day Year}
% Volume and Year of the issue
\publication{N}{YYYY}

%% Information provided by the Editors:
% Issue Title 
\issue{Title of the interdisciplinary issue}
% List of k Editors
\editors{}

\begin{document}

\maketitle

\abstract{The spatial distribution of human settlements into territories strongly conditions their sustainability. Recent forms of urbanization, in particular integrated mega-city regions, may imply different patterns of economic and transportation flows and thus exhibit various performances regarding different indicators of sustainability. This paper proposes to reconstruct endogenous urban regions from the bottom-up using network percolation, in a multi-dimensional way to take into account both urban form (spatial distribution of population) and urban function (properties of transportation networks). Variable parametrizations allows to consider patterns of optimization for two stylized contradictory sustainability indicators (economic performance and greenhouse gases emissions). This suggests a customizable spatial design of policies to develop sustainable territories.}

\keywords{Road network; multilayer percolation; mega-city region}

\section{Introduction}

\strut
\vspace{-4ex}



% les formes des établissements humains peuvent se percevoir a differentes echelles et selon differentes dimensions. Elle font partie de notre vie courante, a ces mutiples echelles, et pourtant nous sont si etrangeres. Afin d'introduire notre sujet, je propose ici une image familière pour bien d'entre vous qui fréquentent cette région.
% La metropole etendue de la region marseilleise ne se limite pas a l'aspect administratif d'aix marseille, mais englobe bien avignon et la ciotat, voire manosque, des lors que l'on prend un certain recul geographique.
% il n'aura pas manque aux eoils attentifs que ce recul est d'autant plus une abstraction, puisque nous ne represnetons ici uniquement les reseaux routiers, qui s'averent deja etre un proxy suffisant pour estrapoler une densite d'activités humaines.
% Cette presentation s'axera ainsi sur la morphologie des reseaux et des territories et sur leurs relations.

The structure of road networks both contains its past growth dynamics and has a significant impact on the sustainability of territories it irrigates. Diverse methods to characterize the structure of spatial networks, and more particularly road networks, have been developed in that context.



\cite{lagesse2015spatial} % definition of the object "way"

% La caracterisation des reseaux routiers, par exemples en termes de classification et dindicateurs topoliques, est liee a differents enjeux qui incluent la comprehnsion des dynamiques ppasses par la contingence de ces artefacts, amis qussi des questions futues liees a la soutenabilité des territoires qui'ils irriguent.
% les developpements recents, issus d'efforts interdispclinaires, incluent par exmeple ce travail de Claire Lagesse applique a la ville que v ous reconnaissaez bien, pour la definition de nouveau object a la pertineence thematique plus grande que les seuls entites des bases de donnes sans ontologie propre. (ex perso ballade avignon hier : vraie coherence thematique de l'objet "Way")




A method to characterize topologies of these spatial networks is network percolation. Percolation in physics can be understood in a broad sense processes related to the progressive occupation or connection of nodes of a network, and is generally associated to a phase transition with the emergence of a giant cluster at a given connection probability \citep{stauffer2014introduction}. Important applications include the quantification of network robustness \citep{callaway2000network}.


Such approaches have been applied to urban systems. \cite{makse1998modeling} model urban growth with a local percolation model. It is also a method applied to the analysis of street networks, for example to extract endogenous urban regions \citep{arcaute2016cities}. In spatial statistics, this method can be used to characterize the spatial morphology of point patterns \cite{huynh2018characterisation}.


\cite{piovani2017urban} % meso scale percolation (within London area) and link with a retail location model

% from monodimensional to multidimensional


%\textit{Towards complementary dimensions to condition road network percolation}
%$\rightarrow$ similar to \cite{cottineau2018defining} to define urban areas


% necessity of multi-dimensional percolation

Existing heuristics however generally focus on a single morphological dimension of networks, and leave out the functional properties of urban systems \citep{burger2012form}. However, urban systems are known to be multidimensional.

This paper addresses such a gap by introducing a multi-dimensional percolation heuristic, in order to combine different dimensions of the urban system, the same way that \cite{cottineau2018defining} combines population density and commuting flows to produce multiple definitions of urban areas.

The rationale behind combining urban form with network topology measures relies on the capture of the link between urban form and function, assumed as distributed by transportation networks \citep{raimbault2018caracterisation}.


% - why the relation between form and function, aka pop and network in our view is crucial
% - take into account in percolation
% - potential application : endogenous sustainable urban entities

% notre approceh reponde a la question ouverte du lien entre ofrme et fonction da`ns les systemes urbaine ;  et par ailleurs se base sur les interatcions entre reseaux te territoires poour fournire un proy de ce lien.
% nous propospns ainsi d'explorer une heuristique multi-dimensionnelle de la percolation qui prend en compte morplholohie urbaine et yopologie dy reseau routier ; nous l'appliquon pour caracterisere endogenement les regions urbaines.
% [rq : deux niveau de reseuax dans notree apporche : (mais en. fait comme souvent en geo ; ) le physique, et le abstraot que l'on construite a partir d'entites spatiales.


%\textbf{Research objective : } \textit{Investigate a multi-dimensional percolation of territorial networks taking into account urban morphology and road network topology; endogenous characterization of urban regions.}


Our contribution relies on several points: (i) this is to the best of our knowledge the first time a multi-dimensional percolation method is appplied to urban systems; (ii) we furthermore apply it on a significant spatial extent; and (iii) we link the cluster obtained with simple sustainability measures.


The rest of this paper is organized as follows: we first describe the multi-dimensional percolation heuristic, the data and variables to which it is applied, and the indicators used to characterized the sustainability of clusters produced. We then describe the results of applying this method to population and network variables for the whole European Union, focusing on the endogenous regions produced and their sustainability properties. We finally discuss possible developments and the implications of this methodology to the design of policies.



%%%%%%%%%%%%%%%%%
\section{Methods}
%%%%%%%%%%%%%%%%%




%%%%%%%%%%%%%%%%%
\subsection{Multi-dimensional percolation}


%%%%%%%%%%%%%%%%%%%%
\begin{figure*}[ht] 
%\resizebox{10cm}{7cm}
  {\includegraphics[width=\linewidth]{figures/principle.pdf}}
  \centering
  \label{fig:method}
  \caption{Schematic representation of the multi-dimensional network percolation heuristic.}
\end{figure*}
%%%%%%%%%%%%%%%%%%%%



Given discrete spatial fields, site percolation is operated between two cells given a threshold parameter for each dimension and a distance threshold. This heuristic is similar to multilayer network percolation \citep{boccaletti2014structure}.

\cite{son2012percolation} % extension of epidemic spreading

\cite{hackett2016bond} bond percolation multiplex networks

% decrivons a presons de neniere stylisee l'algorithme utilise.
% le processus de peroclation initrial est deterministe, base sur la distance, met est prochess des random eulidian networks evoque hier par Alain Franc : si deux neouds sont a une distance d'un rayon critique r_0, ceux ci sont connectes par un lien.
% nous ajoutons alors sur chaque couche du reaseu un seul de peroclation qui determine si le neod peut etre prus en compte dqns le reseau de la couhce.
% par conjonction des contraintes, nous obtenus un reseau a unique couche dans laquelle simplement els composantes connexes donnent les clusters perfcolés.

The method works with an arbitrary number of layers.

The rationale




The parameters implied in this heuristic are the percolation radius $r_0$ and the percolation thresholds $\theta_i$ for each layer $i$.



\subsection{Empirical data}


We apply the heuristic to urban morphology and road network topology measures in Europe. More precisely, a grid with resolution 50km of population density morphology indicators and road network topology indicators, has been computed on spatial moving windows for all European Union by \cite{raimbault2018urban}.

The percolation on such an abstract network is a necessary condition in our case to link the different dimensions considered, namely population distribution and local road network properties.

We percolate the population density layer with a network characteristic layer, that we test among number of edges, number of vertices, cyclomatic number and euclidian efficiency, which capture functional properties especially for the two last.

% Precisons a ppresent les objets geographiques a partir  desqueles nous allons construire le reseau.
% nous rappelns que nous cherchons a combiner prorpiete des territoires, que nous approximerons tres simplement par la districbution spatiale de la population, et propriete des reseaux routiers, que nous approximerons par des indicateirs topollogiques locaux.
% ces differentes caracteristqiues ont ete calucles, dans le cadre de ce travail de modelisation a l'echelle mesocscopique de la coevolution entre reseaux et cilles, sur les feentres glissantes de taille 50km poru l'ensemble de l'europe.
% nous en donnons ici l'illustration dans le cas de la France
% [several infos on different regimes etc]





%%%%%%%%%%%%%%%%
\subsection{Sustainability indicators}


We use this endogenous definition of regional urban systems produced by the percolation algorithm to evaluate their sustainability, in terms of conflicting objectives of economic integration and greenhouse gases emissions. The definition of sustainability, or sustainable development, is by essence multi-dimensional \citep{viguie2012trade}. Its characterization as quantitative indicators is even more subject to numerous degrees of freedom. We work here with two stylized indicators for two conflicting dimensions, as a proof-of-concept.


The EDGAR database \citep{janssens2017edgar} (version 4.3.2) is used for local estimates of greenhouse gases emissions.


% explain indicators : convex hull.

Applying a gravity model to each region, we estimate abstract transportation flows within each and extrapolate emissions by coupling with the Edgar emission database \citep{janssens2017edgar} and economic activities with a scaling law of population.  More precisely, greenhouse gases emissions derived from economic and transportation flows are estimated with the following expression 

\begin{equation}
\phi_{ij} = \left(\frac{v_i v_j}{(\sum_k v_k)^2}\right)^\gamma \cdot \exp\left(\frac{-d_{ij}}{d_0}\right)
\end{equation}

where $v_k$ are either effective local GHG emissions or population. Indeed, the economic activity follows relatively well scaling laws of populations \cite{bettencourt2007growth}, the exponent being dependant on the activity and the definition of areas on which it is estimated~\citep{cottineau2017diverse}.


The sum of all flows within the geographical span of the cluster (that we approximate as the convex Hull envelope of its points), allows us to approximate the cumulated potential emissions and economic activity.


The corresponding relative indicators are defined as
% TODO relative indiactors / normalized (by pop ?)


Using these potential flows follows the logic of \cite{arbabi2019development} which shows a need for improved intra-city-region mobility in England and Wales. Considering the regions as entities in which such transportation development policies can more easily been developed, we look at the sustainability of different possible regions if these potential flows were realized. Varying the parameters $\gamma$ and $d_0$ allows to control for the economic activity considered (high $\gamma$ values correspond to high added-value activities) and the span of interactions through $d_0$.



\section{Results}

\subsection{Implementation}


%Network construction

%We percolate the population density layer with a network characteristic layer, that we test among number of edges, number of vertices, cyclomatic number and euclidian efficiency, which capture functional properties especially for the two last.

% a partir de ces deux couches nous considerosn d'un [art la densite de population, d'autre part des proxus pour la densite ou perdormance du reseau ; avce les paramteres associes.
% la logique geographique derriere cela est liee a la premiere loi de Tobler : deux lieus proches etc  ; ajoute a la logique de la masse (loi de gravite), mais selon different aspects du reseau (car difficile de savoir quelle dimenion a priori jouera le role "equivalent" d'une densite de population : autant laisser libre et explorer plutot que de fixer arbitrairemenbt - maxime d'un computational scientist converti, bien sur.
% en pratqiuem nous contruison le reseau correspondant et isolns ses composantes connexes.
% descirption du plan d'explericence.

%\textbf{Two layers: } population density (threshold $\theta_P$) and network characteristics (threshold $\theta_N$) taken among \{Number of edges, Number of vertices, Cyclomatic number $\mu$, Euclidian efficiency $v$ \}; percolated with a radius $r_0$

\cite{banos2012towards} network efficiency

%\textbf{Rationale: } \textit{two locations will be in relation if they are close, have a high population density and given network characteristics.}


In practice, the analysis is implemented using R and the igraph package. Source code, data and results are available on the open git repository of the project at \url{https://github.com/JusteRaimbault/UrbanMorphology}.



The network is constructed by superposing the population density layer with the network layer, starting from the 5km resolution spatial fields for morphological and network indicators. This network is filtered with the threshold parameters for each layer and with the radius parameter. Connected components yield the clusters that we interpret as endogenous regions.


The experience plan is a full grid, for parameters $r_0$, $\theta_P$, $\theta_N$ and the network variable considered.
We also make $\gamma$ and $d_0$ vary.

We recall that the euclidian performance of the network is in our case $<d_e/d_n>$ where the average is taken on all origin-destination pairs in the network, $d_e$ is the euclidian distance and $d_n$ the network distance. Thus, it indeed increases with network performance, in consistence with the use done here through thresholding.


We systematically explore the clusters obtained for 4800 parameter configurations.





%%%%%%%%%%%%%%%%%%%%
\subsection{Extracting endogenous mega-city regions}


%%%%%%%%%%%%%%%%%%%%
\begin{figure*}[ht] 
%\resizebox{10cm}{7cm}
{\includegraphics[width=0.49\textwidth]{figures/totalPop4183694_00056402_ecount850_radius8000.png}}\\
  {\includegraphics[width=0.49\linewidth]{figures/totalPop2219780_36719597_vcount378_radius8000.png}}
  {\includegraphics[width=0.49\linewidth]{figures/totalPop1474347_36891685_vcount595_radius50000.png}}
  \centering
  
  \caption{Examples of obtained clusters for different parameter values. In the third case for example, we obtain the urban regions of West midlands and London in the UK, Randstad merged with Rhein-Rhur and Rhein-Main in Germany, Paris in France, also with capital cities such as Copenhaguen, Stockholm and Helsinki. There is no cluster in South Europe in that case, due to the high population density threshold.\label{fig:exclusters}}
\end{figure*}
%%%%%%%%%%%%%%%%%%%%

We obtain different endogenous morphologies.

Maps reveal that some configurations resemble the actual distribution of European mega-city regions, which are functionally integrated polycentric urban areas \citep{hall2006polycentric}. These are here defined endogenously from the bottom-up and have a priori no reason to coincide with these functional regions.

We show some examples in Fig.~\ref{fig:exclusters}.




%%%%%%%%%%%%%%%%%%%
\subsection{Percolation transition and fractal dimension}


%%%%%%%%%%%%%%%%%%%%
\begin{figure*}[ht] 
%\resizebox{10cm}{7cm}
  %{\includegraphics[width=\linewidth]{figures/}}
  \centering
  \label{fig:percolation}
  \caption{Percolation transition.}
\end{figure*}
%%%%%%%%%%%%%%%%%%%%


In its application to road networks by \cite{arcaute2016cities}, the structure of the national urban system for UK is captured by studying the percolation transition, i.e. the variation of the size of the largest cluster as a function of the percolation radius. As this signature is tightly linked to historical, cultural and geographical conditions, the application to different urban systems should yield different results. We study here this property, for different threshold parameter values. The relative size of the largest cluster is plotted in Fig.~\ref{fig:percolation} as a function of the percolation radius.

This aspect furthermore gives methodological information on multilayer percolation. Indeed, comparing the result with $\theta_N = 0$ (single layer percolation) with $\theta_N = 0.8$





%%%%%%%%%%%%%%%%%%%
\subsection{Pareto fronts for sustainability}


We show therein that different population, network and distance thresholds yield different performances in terms of sustainability.


%\textit{Superposing Pareto front for observed population and emissions, on all clusters.}
%\includegraphics[height=0.8\textheight]{figures/}

%%%%%%%%%%%%%%%%%%%%
\begin{figure*}[ht] 
%\resizebox{10cm}{7cm}
  {\includegraphics[width=\linewidth]{figures/full_effective_pareto.png}}
  \centering
  \label{fig:paretos}
  \caption{Point clouds of region-level indicators, namely population and emissions, for different parametrizations.}
\end{figure*}
%%%%%%%%%%%%%%%%%%%%


% -> full fronts in reserve slide


% Pareto fronts for aggregated indicators}{
%\textit{Aggregated sustainability indicators suggest some configurations are more Pareto efficient (high $\gamma$ regime, activities with high added value).}

%\includegraphics[height=0.8\textheight]{figures/aggreg_paretos_radiuspopthq.png}





%%%%%%%%%%%%%
\subsection{Linking urban morphology and sustainability}


%# Cumulative Proportion  0.7296 0.9650 0.99761 1.00000
%                PC1       PC2         PC3         PC4
%moran   -0.3088585 0.9493848 -0.04444327  0.03605266
%avgdist  0.5417362 0.1415668 -0.82239570 -0.10072759
%entropy  0.5108424 0.2140447  0.45847647 -0.69499942
%slope    0.5917502 0.1811415  0.33390034  0.71100630

When considering the aggregated indicators for a parametrization of endogenous city regions, one can relate them to morphological indicators for population density spatial distribution, computed by~\cite{raimbault2018calibration}, that we average on areas. This establishes a link between urban morphology and sustainibility. A principal component analysis on considered points yield 96\% of variance with two components, and 73\% explained by the first component alone. The first component relates to a level of monocentricity ($PC1 = -0.3\cdot I + 0.54 \cdot \bar{d} + 0.51\cdot \varepsilon + 0.59 \cdot \alpha$).


%%%%%%%%%%%%%%%%%%%%
\begin{figure*}[ht] 
%\resizebox{10cm}{7cm}
  {\includegraphics[width=\linewidth]{figures/aggreg_morpho_pc1-emissions_targeted.png}}
  \centering
  \caption{Aggregated values of normalized potential emissions, as a function of the first morphological principal component (PC1), for varying values of parameters $d_G$ (rows) and $\gamma_G$ (columns). Other intermediate values for these parameters yield similar behaviors. As PC1 is mainly linked to monocentricity, there seems to exist an optimal intermediate level of monocentricity for emissions alone.\label{fig:emissions-pc1}}
\end{figure*}
%%%%%%%%%%%%%%%%%%%%


As shown in Fig.~\ref{fig:emissions-pc1}, there seems to exist an optimal intermediate level of monocentricity regarding the normalized indicator for emissions only, except for long-range and low-hierarchy interactions.



%%%%%%%%%%%%%%%%%%%%
\begin{figure*}[ht] 
%\resizebox{10cm}{7cm}
  {\includegraphics[width=\linewidth]{figures/aggreg_morpho_relemissions-relefficiency_colpc1_logscale_targeted.png}}
  \centering  
  \caption{Relative potential emissions against relative potential economic unefficiency (both indicators should be minimized), for varying values of $\gamma_G$ (columns) and $d_G$ (rows). Color level gives the value of PC1, whereas point size gives the share of total population contained within considered areas.\label{fig:paretos-relative}}
\end{figure*}
%%%%%%%%%%%%%%%%%%%%


However, when considering both emissions and economic indicators, urban form then acts as a compromise variable, moving points on Pareto fronts, as shown in Fig.~\ref{fig:paretos-relative}. In some case, highly monocentric areas can be a good compromise, whereas the intermediate optimal for emissions may yield highly inefficient areas. This unveils morphological trad-offs, confirming that there is no optimal urban form, but different compromises regarding the conflicting indicators.




%%%%%%%%%%%%%%%%%
\section{Discussion}



%%%%%%%%%%%%%%%%%
\subsection{Developments}

Further work can consist in the use of calibration heuristics to find in a more robust way optimal parameter values. The OpenMOLE model exploration platform provides a transparent access to genetic algorithms for multi-objective optimization \citep{reuillon2013openmole}. The use of such calibration algorithms would allow to unveil the effective form of Pareto fronts, that we may have missed here through the grid sampling.


% -> depending on results


%Grid sampling to explore regions rapidly limited


%\textbf{Implications}

%$\rightarrow$ 


% detail method to extrapolate parameters of gravity flows

% interesting result : compare these emission with the synthetic ? (then just a refinment ?) -> yes, better. show new Pareto fronts with this fixed parameters for only emissions, and varying for economic ? 



Extrapolating transportation flows with a spatially explicit gravity and flow model can allow to compare these with actual transportation flows in the emission database, and yield a possible calibration. These extrapolated parameters could then be used within the economic and emissions potentials.

An other potential development would imply crossing our endogenous definitions of urban regions with socio-economic databases, and compute indicators implied in other dimensions of sustainability, for example related to socio-economic inequalities, spatial distribution of accessibilities, activities with different scaling exponents.
% ex : at least different scaling exponent.



%%%%%%%%%%%%%%%%%
\subsection{Towards policy applications}


This suggests policies in terms of regional integration to increase the sustainability of mega-city regions.

The way such results can be transferred to policy-making recommandations remains an open question, but Pareto-optimal configurations can be used for the planning of regional transportation networks for example, or to design policies for the distribution of subsidies.

%%%%%%%%%%%%%%%%%
\section{Conclusion}

In conclusion, our multilayer percolation approache captures in a way the multi-dimensionality of urban systems and a link between form and function.


%$\rightarrow$ Empirical and theoretical research directed towards concrete policy-making applications. \textbf{Need for more data-driven approaches.}
%First step towards systematic benchmarks and multi-modeling. \textbf{Need for more systematic model exploration.} % model coupling / benchmarking

%$\rightarrow$ Towards multi-scalar approaches ? \textbf{Need for more integrated models.}
 %Model integration and multi-scalarity ? \textbf{Need for more integrated models.}

%$\rightarrow$ Multidimensionality of urban systems ? \textbf{Need for more interdisciplinarity.}









\bibliographystyle{jimis-en}
\bibliography{biblio}



%\appendix\footnotesize

%\section{Appendix 1: supplementary sensitivity analyses}



%\sframe{Results: effective emissions}{

%\textit{Effective emissions exhibit a supralinear scaling of population}
%\includegraphics[height=0.83\textheight]{figures/aggreg_effective.png}

%\sframe{Results: all clusters Pareto fronts}{

%\textit{Variation of Pareto front patterns when potential parameter $\gamma,d_0$ vary.}

%\includegraphics[width=0.9\textwidth]{figures/full_paretos.png}


%\sframe{Results: an optimal morphology}{

%\textit{More monocentric areas are more optimal in terms of relative emissions and efficiency ?}

%\includegraphics[width=0.49\textwidth]{figures/aggreg_morpho_pc1-relefficiency.png}
%\includegraphics[width=0.49\textwidth]{figures/aggreg_morpho_pc1-relemissions.png}

% TODO : curves for emissions rescale to the same -> link with Caruso profiles ? ; investigate that, and why not for efficiency.





%\section{Acknowledgment}
%

%\section{Biography}
%Here, feel free to add short biographies of the authors.






\end{document}





%%%%%%%%
%% -- TEMPLATES

%\begin{table}
%  \newcolumntype{+}{>{\global\let\currentrowstyle\relax}}
%  \newcolumntype{^}{>{\currentrowstyle}}
%  \newcommand{\rowstyle}[1]{\gdef\currentrowstyle{#1}%
%    #1\ignorespaces
%  }
%  \centering
%  \begin{tabular}{+>{\bfseries}l^c^c^c^c}
%    \hline
%    \rowstyle{\bfseries}
%    & Sepal.Length & Sepal.Width & Petal.Length & Petal.Width\\
%    Setosa & 5.006 & 3.428 & 1.462 & 0.246\\
%    Versicolor & 5.936 & 2.77  & 4.26  & 1.326\\
%    Verginica & 6.588 & 2.974 & 5.552 & 2.026\\
%    \hline
%  \end{tabular}
%  \caption{Morbi malesuada diam at magna condimentum.}
%  \label{tab:example}
%\end{table}

%
%\begin{figure*}[ht] 
%\resizebox{10cm}{7cm}
%  {\includegraphics[width=11cm]{spider.jpg}}
%  \centering
%  \label{frog}
%  \caption{A spider, Picture from Didier Josselin.}
%  \end{figure*}
%  


